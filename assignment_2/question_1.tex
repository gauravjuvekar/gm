\documentclass[a4paper,14pt,english,crop=false]{standalone}
\usepackage{assignment}
\begin{document}
\section{What is aliasing and what are antialiasing techniques?}
Aliasing is a distortion observed due to a lower sampling rate (display
resolution) of an image. It causes artefacts such as jagged edges of a smooth
curve or line, loss of detail of an image, or the appearance of regular patterns
win the rendered image.
Some antialiasing techinques are:
\begin{description}
  \item [Preflitering] {computes a pixel colour based on the overlap of the
      scene objects with the pixel's area. This causes a blurring effect along
      line edges and reduces jagged edges which reduces aliasing effect.}
  \item [Postfiltering or supersampling] {renders an image at a higher
      resolution, lowpass filters it and the image is then resampled at the
      required resolutions.  This is computationally expensive as more memory is
      required for rendering at a higher resolution.}
  \item [Adaptive supersampling] {is a variant of supersampling which samples at
      a higher resolution only where the pixel colours change rapidly.}
\end{description}
\end{document}
