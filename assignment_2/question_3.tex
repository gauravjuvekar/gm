\documentclass[a4paper,14pt,english,crop=false]{standalone}
\usepackage{assignment}
\begin{document}
\section{Explain polygon drawing and filling algorithms.}
Polygons can be rendered by drawing the edges using any line drawing algorithm
like DDA or Bresenham's algorithm.
Polygon filling involves colouring a bounded region (polygon) with a given
colour. There are various algorithms to determine whether a pixel is within or
outside the region.\\
\begin{figure}[h]
  \centering
  \includesvg[width=0.5\textwidth,pdf]{scanline}
  \caption{Scanline fill}
\end{figure}
Some polygon filling algorithms are
\begin{description}
  \item [Scanline fill] {iterates a scanline across the image (horizontal or
      vertical) and colours the pixels of the line between alternating
      intersections with the polygon boundaries.}
  \item [Flood fill] {starts with a point in the interior of
      the polygon and colours all neighbouring pixels of the same target colour
      with a replacement colour. This is repeated till all the pixels in the
      interior of the region are coloured. There are 4-connected and
      8-connected variants depending on which pixels are considered to be
      neighbours at each step of the algorithm.}
  \item [Boundary fill] {is very similar to flood fill, but the algorithm stops
      when a boundary of a given colour is reached.}
\end{description}
\end{document}
