\documentclass[a4paper,14pt,english,crop=false]{standalone}
\usepackage{assignment}
\begin{document}
\section{Explain polygon drawing and filling algorithms.}
Polygons can be rendered by drawing the edges using any line drawing algorithm
like DDA or Bresenham's algorithm.
Polygon filling involves colouring a bounded region (polygon) with a given
colour. There are various algorithms to determine whether a pixel is within or
outside the region.
Some polygon filling algorithms are
\begin{description}
  \item [Scanline fill] {iterates a scanline across the image (horizontal or
      vertical) and colours the pixels of the line between alternating
      intersections with the polygon boundaries.}
  \item [Flood fill or boundary fill] {starts with a point in the interior of
      the polygon and colours all neighbouring pixels with the target colour if
      they are not a part of the boundary. This is repeated till all the pixels
      in the interior of the region are coloured. There are 4-connected and
      8-connected variants depending on which pixels are considered to be
      neighbours at each step of the algorithm.}
\end{description}
\end{document}
