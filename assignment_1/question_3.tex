\section{What is rendering? What are different rendering techniques?}

Rendering is the process of generating an image from a 2D or 3D model using a
computer program. The model defines the viewpoint, shapes, colours of objects,
lighting properties of the scene, properties of object textures like specular or
diffuse refletivity, etc. A renderer generates a digital image from this data.

Some different rendering techniques are
\begin{description}
  \item [Ray tracing]traces the path of light through pixels in an image plane
    and simulates the effects when the light ray interacts with objects in the
    scene. This can produce a highly realistic image of a scene, however, the
    algorithm is complex and rendering takes a long time for largeer scenes.
  \item [Radiosity] simulates how illuminated surfaces act as indirect light
    sources which in turn illuminate other surfaces. This produces a realistic
    shading of scenes with ambient lighting, suitable for indoor scenes.
  \item [Rasterisation] updates pixels affected by each scene primitive (like
    triangles and polygons) to obtain a final image.
  \item [Ray casting] is a primitive form of ray tracing which considers only
    primary rays, ignoring the way light is affected by secondary reflections or
    interactions.
\end{description}
